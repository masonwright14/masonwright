\documentclass[letterpaper,12pt]{article}
\usepackage{amsmath}
\author{Mason Wright}
\title{Homework 9}
\begin{document}
\maketitle
$\mathbf{1.i}$
\\True.
\begin{equation*}
\| \vec{u} + \vec{v} \| ^2 = ( \vec{u} + \vec{v} ) \cdot ( \vec{u} + \vec{v} ) = \vec{u} \cdot \vec{u} + 2 \vec{u} \cdot \vec{v} + \vec{v} \cdot \vec{v}
\end{equation*}
\begin{equation*}
\| \vec{u} \| ^2 + \| \vec{v} \| ^2 = \vec{u} \cdot \vec{u} + \vec{v} \cdot \vec{v}
\end{equation*}
Therefore,
\begin{equation*}
\| \vec{u} + \vec{v} \| ^2 = \| \vec{u} \| ^2 + \| \vec{v} \| ^2 \rightarrow 2 \vec{u} \cdot \vec{v} = 0 
\end{equation*}
Thus, $\vec{u}$ and $\vec{v}$ are orthogonal if $\| \vec{u} + \vec{v} \| ^2 = \| \vec{u} \| ^2 + \| \vec{v} \| ^2$.
\\
\\$\mathbf{1.ii}$
\\False. Counterexample:
Let $\vec{u} = 
\begin{vmatrix}
1 \\
0
\end{vmatrix}$, and let
$\vec{v} =
\begin{vmatrix}
0 \\
2
\end{vmatrix}$.
\begin{equation*}
\vec{u} + \vec{v} = 
\begin{vmatrix}
1 \\
2
\end{vmatrix}
\end{equation*}
\begin{equation*}
\vec{u} - \vec{v} = 
\begin{vmatrix}
1 \\
-2
\end{vmatrix}
\end{equation*}
\begin{equation*}
\| \vec{u} + \vec{v} \| = \| \vec{u} - \vec{v} \| = \sqrt{5}
\end{equation*}
\begin{equation*}
\| \vec{u} \| = 1 \neq \| \vec{v} \|
\end{equation*}
\\
\\$\mathbf{1.iii}$
\\True. Any $\vec{w} \in W^{\perp}$ must be in $V^{\perp}$, because any $\vec{w} \in W^{\perp}$ is orthogonal to all vectors in
$W$, including all vectors in $V$.
\\
\\$\mathbf{1.iv}$
\\True. Any $\vec{w} \in W$ must be in $V$, because any $\vec{w} \in W$ is orthogonal to all vectors in $W^{\perp}$, 
which includes all vectors in $V^{\perp}$, which means that all $\vec{w} \in W$ are in $(V^{\perp})^{\perp}$, which is just $V$.
\\
\\$\mathbf{1.v}$
\\False. Counterexample: Let $V$ be the span of $\begin{vmatrix} 1 \\ 0 \end{vmatrix}$, and let $W$ be the span of
$\begin{vmatrix} 0 \\ 1 \end{vmatrix}$, with $V$ and $W$ in $\mathbf{R}^2$.
Then $V^{\perp}$ is the span of $\begin{vmatrix} 0 \\ 1 \end{vmatrix}$, and $W^{\perp}$ is the span of 
$\begin{vmatrix} 1 \\ 0 \end{vmatrix}$. Thus, $V^{\perp} \cap W^{\perp} = \{ \vec{0} \}$.
So $(V^{\perp} \cap W^{\perp})^{\perp}$ is all of $\mathbf{R}^2$. All of $\mathbf{R}^2$ is not contained in
$V \cup W$, so the statement is false.
\\
\\$\mathbf{2.i}$
\\
$\vec{v_1} =
\begin{vmatrix}
1 \\
2 \\
0 \\
2
\end{vmatrix}$, so $\| \vec{v_1} \|$ is $3$. Thus, $\vec{u_1} =
\begin{vmatrix}
1/3 \\
2/3 \\
0 \\
2/3
\end{vmatrix}$
\\$\vec{v_2}^{\perp} = \vec{v_2} - (\vec{v_2} \cdot \vec{u_1}) \vec{u_1}$, so:
\begin{align*}
\vec{v_2}^{\perp} & =
\begin{vmatrix}
2 \\
2 \\
-1 \\
6
\end{vmatrix}
- \Bigg(
\begin{vmatrix}
2 \\
2 \\
-1 \\
6
\end{vmatrix}
\cdot
\begin{vmatrix}
1/3 \\
2/3 \\
0 \\
2/3
\end{vmatrix}
\Bigg)
\begin{vmatrix}
1/3 \\
2/3 \\
0 \\
2/3
\end{vmatrix} \\
& =
\begin{vmatrix}
0 \\
-2 \\
-1 \\
2
\end{vmatrix}
\end{align*}
\begin{equation*}
\vec{u_2} =
\vec{v_2}^{\perp} / \| \vec{v_2}^{\perp} \| =
\vec{v_2}^{\perp} / 3 =
\begin{vmatrix}
0 \\
-2/3 \\
-1/3 \\
2/3
\end{vmatrix}
\end{equation*}
\begin{align*}
\vec{v_3}^{\perp} & = \vec{v_3} - (\vec{v_3} \cdot \vec{u_1}) \vec{u_1} - (\vec{v_3} \cdot \vec{u_2}) \vec{u_2}
\\ & = 
\begin{vmatrix}
5 \\
3 \\
1 \\
-1
\end{vmatrix}
- \Bigg(
\begin{vmatrix}
5 \\
3 \\
1 \\
-1
\end{vmatrix}
\cdot
\begin{vmatrix}
1/3 \\
2/3 \\
0 \\
2/3
\end{vmatrix}
\Bigg)
\begin{vmatrix}
1/3 \\
2/3 \\
0 \\
2/3
\end{vmatrix}
- \Bigg(
\begin{vmatrix}
5 \\
3 \\
1 \\
-1
\end{vmatrix}
\cdot
\begin{vmatrix}
0 \\
-2/3 \\
-1/3 \\
2/3
\end{vmatrix}
\Bigg)
\begin{vmatrix}
0 \\
-2/3 \\
-1/3 \\
2/3
\end{vmatrix}
\\ & =
\begin{vmatrix}
4 \\
-1 \\
0 \\
-1
\end{vmatrix}
\end{align*}
\begin{equation*}
\vec{u_3} = \vec{v_3}^{\perp} / \| \vec{v_3}^{\perp} \| 
=
\vec{v_3}^{\perp} / \sqrt{18}
= 
\begin{vmatrix}
2 \sqrt{2} / 3 \\
-\sqrt{2} / 6 \\
0 \\
-\sqrt{2} / 6
\end{vmatrix}
\end{equation*}
\\
\\$\mathbf{2.ii}$
\\ Q is 
$\begin{bmatrix}
\vec{u_1} & \vec{u_2} & \vec{u_3}
\end{bmatrix}$, or 
$\begin{bmatrix}
1/3 & 0 & 2 \sqrt{2} / 3 \\
2/3 & -2/3 & - \sqrt{2} / 6 \\
0 & -1/3 & 0 \\
2/3 & 2/3 & -\sqrt{2} / 6
\end{bmatrix}$.
\\R is 
$\begin{bmatrix}
\| \vec{v_1} \| & \vec{u_1} \cdot \vec{v_2} & \vec{u_1} \cdot \vec{v_3} \\
0 & \| \vec{v_2}^{\perp} \| & \vec{u_2} \cdot \vec{v_3} \\
0 & 0 & \| \vec{v_3}^{\perp} \|
\end{bmatrix}$.
\\ $\| \vec{v_1} \| = 3$. $\| \vec{v_2}^{\perp}  \| = 3$. $\| \vec{v_3}^{\perp}  \| = \sqrt{18}$.
\begin{equation*}
\vec{u_1} \cdot \vec{v_2} =
\begin{vmatrix}
1/3 \\
2/3 \\
0 \\
2/3
\end{vmatrix}
\cdot
\begin{vmatrix}
2 \\
2 \\
-1 \\
6
\end{vmatrix}
= 6
\end{equation*}
\begin{equation*}
\vec{u_1} \cdot \vec{v_3} =
\begin{vmatrix}
1/3 \\
2/3 \\
0 \\
2/3
\end{vmatrix}
\cdot
\begin{vmatrix}
5 \\
3 \\
1 \\
-1
\end{vmatrix}
= 3
\end{equation*}
\begin{equation*}
\vec{u_2} \cdot \vec{v_3} =
\begin{vmatrix}
0 \\
-2/3 \\
-1/3 \\
2/3
\end{vmatrix}
\cdot
\begin{vmatrix}
5 \\
3 \\
1 \\
-1
\end{vmatrix}
=
-3
\end{equation*}
\begin{equation*}
R =
\begin{bmatrix}
3 & 6 & 3 \\
0 & 3 & -3 \\
0 & 0 & 3 \sqrt{2}
\end{bmatrix}
\end{equation*}
\\
\\$\mathbf{2.iii}$
\begin{equation*}
proj_V(\vec{x}) = (\vec{u_1} \cdot \vec{x}) \vec{u_1} +  (\vec{u_2} \cdot \vec{x}) \vec{u_2} +  (\vec{u_3} \cdot \vec{x}) \vec{u_3}
\end{equation*}
\begin{equation*}
A = 
\begin{bmatrix}
proj_V(\vec{e_1}) & 
proj_V(\vec{e_2}) & 
proj_V(\vec{e_3}) & 
proj_V(\vec{e_4})
\end{bmatrix}
\end{equation*}
\begin{equation*}
proj_V(\vec{e_1})  =
1/3 \vec{u_1} + 2 \sqrt{2} / 3 \vec{u_3} =
\begin{vmatrix}
1 \\
0 \\
0 \\
0
\end{vmatrix}
\end{equation*}
\begin{equation*}
proj_V(\vec{e_2})  =
2/3 \vec{u_1} - 2/3 \vec{u_2} - \sqrt{2}/6 \vec{u_3} =
\begin{vmatrix}
0 \\
17 / 18 \\
2 / 9 \\
1 / 18
\end{vmatrix}
\end{equation*}
\begin{equation*}
proj_V(\vec{e_3})  =
-1/3 \vec{u_2} =
\begin{vmatrix}
0 \\
2/9 \\
1/9 \\
-2/9
\end{vmatrix}
\end{equation*}
\begin{equation*}
proj_V(\vec{e_4})  =
2/3 \vec{u_1} + 2/3 \vec{u_2} - \sqrt{2}/6 \vec{u_3} =
\begin{vmatrix}
0 \\
1/18 \\
-2/9 \\
17/18
\end{vmatrix}
\end{equation*}
\begin{equation*}
A =
\begin{bmatrix}
1 & 0 & 0 & 0 \\
0 & 17/18 & 2/9 & 1/18 \\
0 & 2/9 & 1/9 & -2/9 \\
0 & 1/18 & -2/9 & 17/18
\end{bmatrix}
\end{equation*}
\\
\\$\mathbf{3.i}$
\\ $T(\vec{x} - T(\vec{x})) = T(\vec{x}) - T(T(\vec{x}))$, because $T$ is a linear transformation.
It is given that $T(T(\vec{x})) = T(\vec{x})$.
So $T(\vec{x} - T(\vec{x})) = T(\vec{x}) - T(\vec{x}) = \vec{0}$. Therefore,
$(\vec{x} - T(\vec{x})) \in Ker(T)$.
\begin{align*}
\vec{x} &= \vec{x} + T(\vec{x}) - T(\vec{x}) \\ & =
T(\vec{x}) + (\vec{x} - T(\vec{x}))
\end{align*}
But $T(\vec{x}) \in Im(T)$, and $(\vec{x} - T(\vec{x})) \in Ker(T)$.
So any $\vec{x}$ can be written as the sum of some $\vec{y} \in Ker(T)$, and some $\vec{z} \in Im(T)$.
\\
\\$\mathbf{3.ii}$
\\
\begin{align*}
f(t) & = \| t \vec{y} + \vec{z} \|^2
\\ & = (t \vec{y} + \vec{z}) \cdot (t \vec{y} + \vec{z})
\\ & = t^2 (\vec{y} \cdot \vec{y}) + 2t (\vec{y} \cdot \vec{z}) + \vec{z} \cdot \vec{z}
\\ & = t^2 \| \vec{y} \|^2 + 2t (\vec{y} \cdot \vec{z}) + \| \vec{z} \|^2
\end{align*}
\\
\\$\mathbf{3.iii}$
\\
Because $T$ is a linear transformation and $\vec{y}$ is in the kernel of $T$:
\begin{equation*}
T(t \vec{y} + \vec{z}) = t T(\vec{y}) + T(\vec{z}) = \vec{0} + T(\vec{z}) = T(\vec{z})
\end{equation*}
Because $\| T(  \vec{x} ) \| \leq \| \vec{x} \|$ for all $\vec{x}$, $\| T(  \vec{z} ) \| \leq \| t \vec{y} + \vec{z} \|$.
Therefore, since sizes are non-negative, $\| T(  \vec{z} ) \|^2 \leq \| t \vec{y} + \vec{z} \|^2$.
Because $\vec{z} = T(\vec{x})$ for some $\vec{x}$, $T(\vec{z}) = T(T(\vec{x})) = T(\vec{x}) = \vec{z}$.
So $T(\vec{z}) = \vec{z}$. Therefore, $\| \vec{z} \|^2 \leq \| t \vec{y} + \vec{z} \| ^2$.
\\
\\But $f(t) - f(0) = \| t \vec{y} + \vec{z} \| ^2 - \| \vec{z} \|^2$, since $f(0) = \| \vec{z} \| ^2$. So $f(t) - f(0)$ is always
non-negative, meaning that $f(0)$ is a global minimum.
\\
\\$\mathbf{3.iv}$
\\$f(t)$ is a quadratic function, because $f(t) = a t^2 + b  t + c$, where $a = \| \vec{y} \| ^2$, $b = 2(\vec{y} \cdot \vec{z})$, and 
$c = \| \vec{z} \| ^2$. So because there is a global minimum at $0$, $f'(0) = 0$.
$f'(t) = 2 \|\vec{y}\|^2 t + 2(\vec{y} \cdot \vec{z})$, so $f'(0) = 2 (\vec{y} \cdot \vec{z})$. This means that
$\vec{y} \cdot \vec{z} = 0$, so $\vec{y}$ and $\vec{z}$ are orthogonal.
\\
\\$\mathbf{3.v}$
\\
An orthogonal projection onto vector subspace $V$ is a function $L(\vec{x})$ such that $L(\vec{x}) = \vec{x}^{\parallel}$, where
$\vec{x} = \vec{x}^{\parallel} + \vec{x}^{\perp}$, $\vec{x}^{\perp}$ is orthogonal to all $\vec{v} \in V$, 
and $\vec{x}^{\parallel} \in V$, for any $\vec{x}$ in the domain.
\\\\We have shown that any $\vec{y} \in Ker(T)$ is orthogonal to any $\vec{z} \in Im(T)$, and that $\vec{x} - T(\vec{x}) \in Ker(T)$.
Therefore, $\vec{x} - T(\vec{x})$ is orthogonal to any $\vec{v} \in Im(T)$. We know that any $\vec{x} = (\vec{x} - T(\vec{x})) + T(\vec{x})$. $T(\vec{x}) \in Im(T)$ by definition of the image.
We know that the image of $T$ is a linear subspace, because it is the image of a linear
transformation.
\\
\\Thus, $T$ is the orthogonal projection onto $Im(T)$, where $T(\vec{x}) = \vec{x}^{\parallel}$ and $\vec{x} - T(\vec{x}) = \vec{x}^{\perp}$.
\end{document}